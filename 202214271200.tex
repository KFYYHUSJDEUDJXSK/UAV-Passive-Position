\documentclass{cumcmthesis}
%\documentclass[withoutpreface,bwprint]{cumcmthesis} %去掉封面与编号页

\usepackage{tabularx}
\title{论文题目}
\tihao{C}            % 题号
\baominghao{202217241200}    % 报名号
\schoolname{华中科技大学}
\membera{卢凯}
\memberb{陈铭锐}
\memberc{房怿宽}
\supervisor{胡勇}
\yearinput{2022}     % 年
\monthinput{09}      % 月
\dayinput{15}        % 日

\begin{document}
	\maketitle
	\begin{abstract}
		摘要的具体内容。
		\keywords{关键词1\quad  关键词2\quad   关键词3}
	\end{abstract}
	%\tableofcontents
	
	
%----------- 正文 ----------
%----------- 一、问题重述 ----------
	\section{问题重述}
		\subsection{问题背景}
		\par 
		\subsection{待求解的问题}
		\par 
			\begin{enumerate}
				\item{问题一:}
				\item{问题二:}
				\item{问题三:}
			\end{enumerate}
%----------- 二、问题分析 ----------
	\section{问题分析}
		\subsection{问题一:}
		\subsection{问题二:}
		\subsection{问题三:}
%----------- 三、模型的假设与约定 ----------
	\section{模型的假设与约定}
		\begin{itemize}
			\item{} one
			\item{} two
			\item{} three
		\end{itemize}
%----------- 四、符号说明及名词定义 ----------
	\section{符号说明及名词定义}
		\begin{center}
			\begin{table}[h]
				\caption{本论文所使用的符号}
				\begin{tabularx}{\textwidth}{p{0.08\textwidth}X}
					\toprule	
					
					$D_i$ & 编号为$i$的无人机的理想位置  \\
					$\widehat{D_i}$ & 编号为$i$的无人机的有偏实际位置  \\
					$D_i(\rho,\theta)$ & 用极坐标表示的无人机的位置 \\
					  
					$\overrightarrow{D_iD_k}$ & 从$D_i$指向$D_k$的矢量 \\
					$\omega_i$      & 编号为$i$的无人机的误差矢量,即$\overrightarrow{\widehat{D_i}D_i}$ \\
					$\bigodot O_i$    & 第$i$个理想圆  \\
					
					$hc_k$     & Holding cost of product k per unit and period;holding cost of product k per unit and period;holding cost of product k per unit and period   \\
					$hc_k$     & holding cost of product k per unit and period   \\
					oc   & overtime costs per unit   \\
					$M_{kt}$   & bignumber for product k in period t\\
					$pc_{k}$   & production cost of product k per unit\\
					$pc^r_{k}$ & remanufacturing cost of product k per unit\\
					$sc_{k}$   &  setup cost of product k\\
					$sc^r_{k}$ &  setup cost of returned product k\\
					$tp_{k}$   &  production time for one unit of product k\\
					$tp^r_{k}$ & remanufacturing time for one unit of product k\\
					$ts_{k}$   & setup time of product k\\
					$ts^r_{k}$ & setup time of returned product k\\
					
					 
					$BL_{kt}$  & backlog of product k at the end of period t\\
					$D_{kt}$   & external demand of product k in period t  \\
					$I_{kt}$   & net inventory of product k at the end of period t  \\
					$I^r_{kt}$ & net inventory of returns of product k at the end of period t  \\
					$IP_{kt} $  & physical inventory of product k at the end of period t \\
					$IP^r_{kt}$   & physical inventory of recoverables at the end of period t \\
					$R_{kt}$   &  returns of product k in period t\\
					$SF^r_{kt}$   &  Shortfall of recoverables of product k in t \\
					\bottomrule
				\end{tabularx}
			\end{table}
		
		\begin{tabular}{cc}
				\hline
				\makebox[0.3\textwidth][c]{符号}	&  \makebox[0.4\textwidth][c]{意义} \\ \hline
				D	    & 木条宽度(cm) \\ \hline
		\end{tabular}
		\end{center}
%----------- 五、模型的建立与求解 ----------
	\section{模型的建立与求解}
		\subsection{问题一}
			\subsubsection{问题一模型的建立}
			\subsubsection{问题一的具体求解}
			
		\subsection{问题二}
			\subsubsection{title}
			\subsubsection{title}
			\subsubsection{title}
		\subsection{问题三}
			\subsubsection{title}
			\subsubsection{title}
			\subsubsection{title}
	\section{总结}
	\begin{thebibliography}{9}%宽度9
		\bibitem{bib:one} ....
	\end{thebibliography}
	\begin{appendices}
		附录的内容。
	\end{appendices}
\end{document}